\documentclass[11pt]{article}
\usepackage{amsmath}
\usepackage{amsfonts}
\usepackage{amssymb}
\usepackage{color}
\usepackage{mathpazo}
\usepackage{multicol}
\usepackage{hyperref}
\usepackage{enumitem}
\usepackage{graphicx}
\usepackage{fancyhdr}
\usepackage{verbatim}
\usepackage[margin=.8in]{geometry}

\hypersetup{
    colorlinks=true,
    linkcolor=blue,
    filecolor=magenta,      
    urlcolor=blue,
}

\pagestyle{fancy}
\lhead{\footnotesize University \textit{of} Nebraska-Lincoln}
\chead{Course Syllabus}
\rhead{\footnotesize Math 391}
\setlength{\headheight}{2cm}

\topmargin = -1.25in
\setlist{noitemsep}

 %%%%%%%%%%%%%%%%%%%%%%%%%%%%%%
 %%%%         Please fill out the following block       %%%%%
 %%%%%%%%%%%%%%%%%%%%%%%%%%%%%%
\newcommand{\YourName}[0]{Dr. Levi Heath}
\newcommand{\YourEmailAddress}[0]{lheath2@unl.edu}
\newcommand{\YourPronouns}[0]{he/him}
\newcommand{\YourOffice}[0]{AVH 225}

\begin{document}

\begin{center}
    {\LARGE Math 391: Mathematics of Machine Learning}
\end{center}

%%%%%%%%%%
\section*{Basic Information}
%%%%%%%%%%
\vspace{-.5em}
\begin{description}[nosep]

\item[Instructor] \YourName\  (\YourPronouns) 

\item[Email] \href{mailto:\YourEmailAddress}{{\tt \YourEmailAddress}}

%\item[Office] \YourOffice 

\item[\textcolor{red}{Office Hours}] By appointment--if you have a question about an assignment or would like to meet for any other reason, please send me an email and we can set something up! \\

\item[Class Meeting Time and Location] 9 AM--12 PM MTWRF on Zoom--link on Canvas homepage.

\item[Required Materials] You will need a computer to attend class and complete programming assignments. The library offers laptops, which you can check out for free: \href{https://libraries.unl.edu/laptops}{https://libraries.unl.edu/laptops}.  
\item[Course Website] All information will be on Canvas.

\item[Textbook] Michael Nielsen's {\it Neural Networks and Deep Learning}, which is freely available online: \\
\href{http://neuralnetworksanddeeplearning.com/}{http://neuralnetworksanddeeplearning.com/} \\
We will not follow this book directly, but it is a nice resource if you like referencing textbooks.

\item[Prerequisites] A grade of P, C, or better in Math 107 or Math 107H. \\

\item[Course Description] An introduction to the mathematical underpinnings of machine learning. Topics include matrix algebra, gradient descent, multidimensional arrays, artificial neural networks, and implementation with MATLAB or Python. Additional topics vary but can include support vector machines, decision trees, and large language models.

\end{description}

%%%%%%%%%%
\section*{Grades and Assessments}
%%%%%%%%%%
\vspace{-.5em}
\begin{description}[nosep]
    \item[Grade Breakdown] There will be five components to your grade for this term, weighted as shown below. \vspace{-.5em}
    \begin{center}
        \hspace{-3em}
        \begin{tabular}{|c|c|c|}
             \hline
              Participation & Homework & Projects \\ \hline
              20\%  &  50\% &  30\%  \\ 
              \hline
        \end{tabular}
    \end{center}
    
    
    \item[Letter Grades] Letter grades will be given no stricter than:\vspace{.125em} \\
        \begin{tabular}{|c|c|c|c|c|c|c|c|c|c|c|}
             \hline
              A & A- & B+ & B & B- & C+ & C & C- & D+ & D & F \\ \hline
              90-100\%  &  87-89\%  & 84-86\% & 80-83\% & 77-79\%  & 74-76\%  & 70-73\%  & 67-69\%  & 64-66\%  & 60-63\%  &  0-60\% \\ \hline
        \end{tabular}\vspace{.5em}
    
    \item[Participation] In class, we will often work in groups to complete activities or worksheets. Your participation during class will determine part of your grade.\vspace{.25em}
    
    %\item[Quizzes] We will have in-class quizzes to assess your understanding of the material.\vspace{.25em}
    
    \item[Homework] There will be regular homework assignments to allow you to practice concepts learned in class. Each assignment will be accessible through our Canvas course page.  Many current and past instructors spent a significant amount of time curating these problems for you, so please take advantage of this resource which is a significant percentage of your grade.\vspace{.25em}
    
    \item[Projects] To assess your ability to apply the mathematics we learn, there will be three projects. Two of the projects will involve building machine learning models using Python and TensorFlow (an industry-standard open-source programming package for machine learning). Please note that you are not expected to have any prior programming experience. One of the projects will involve giving an in-class presentation. You will work in groups of two for all of the projects. \vspace{.5em}

    \item[Make-up Work] Make-up exams and homework will be given at the instructor's discretion. If you will miss an assignment for some reason, please let your instructor know so they can do their best to figure something out. Our only goal for the course is for you to learn linear algebra and completing every assignment is crucial to reaching that goal!
\end{description}

%%%%%%%%%%
\section*{Policies}
%%%%%%%%%%
\vspace{-.5em}
\begin{description}[nosep]

\item[Instructional Continuity Plans for when Classes are Canceled] If in-person classes are canceled, you will be notified of the instructional continuity plan by an announcement or email in Canvas.\vspace{.25em}

\item[Departmental Grading Appeals] The Department of Mathematics does not tolerate discrimination or harassment on the basis of race, gender, religion, or sexual orientation. If you believe you have been subject to such discrimination or harassment, in this or any other math course, please contact the department. If, for this or any other reason, you believe your grade was assigned incorrectly or capriciously, then appeals may be made to (in order) the instructor, the vice chair, the department grading appeals committee, the college grading appeals committee, and the university grading appeals committee.\vspace{.25em}


\item[University Policies] Students are responsible for knowing the university policies and resources found on this page (\href{http://go.unl.edu/coursepolicies}{http://go.unl.edu/coursepolicies}):
\begin{itemize}
    \item University-wide Attendance Policy
    \item Academic Honesty Policy
    \item Services for Students with Disabilities
    \item Mental Health and Well-Being Resources
    \item Final Exam Schedule
    \item Fifteenth Week Policy
    \item Emergency Procedures
    \item Diversity \& Inclusiveness
    \item Title IX Policy
    \item Other Relevant University-Wide Policies
\end{itemize}
\end{description}

%%%%%%%%%%
\section*{Tentative Schedule}
%%%%%%%%%%
\vspace{-.5em}

\subsection*{Week 1}
\vspace{-.6em}
\begin{description}[nosep]
    \item[Thursday, Jan. 2] Multivariable Functions and Gradient Descent
    \item[Friday, Jan. 3] Matrix Algebra and Transformations
\end{description}

\subsection*{Week 2}
\vspace{-.6em}
\begin{description}[nosep]
    \item[Monday, Jan. 6] Feedforward Neural Networks and Intro to Python Programming
    \item[Tuesday, Jan. 7] Project 1: How to Train Your Model
    \item[Wednesday, Jan. 8] Finish Project 1 
    \item[Thursday, Jan. 9] Backpropagation 
    \item[Friday, Jan. 10] Training from Scratch
\end{description}

\subsection*{Week 3}
\vspace{-.6em}
\begin{description}[nosep]
    \item[Monday, Jan. 13] Project 2: Getting to know Machine Learning
    \item[Tuesday, Jan. 14] Finish and present Project 2
    \item[Wednesday, Jan. 15] Model Testing
    \item[Thursday, Jan. 16] Project 3: Dataset Benchmarking
    \item[Friday, Jan. 17] Finish Project 3
\end{description}

\end{document}

